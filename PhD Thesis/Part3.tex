\part{Entrepreneurship in a Platform Economy} 
\label{part:entrepreneurshipPlatformEconomy}

\section*{Overview of Part III}

Out of the economic states discussed within the monograph, the Platform economy is the one most closely related to that of contemporary society. Within this economy hierarchical production organisations become more developed and sprawling structures such that platforms are created: hierarchical production organisations that exists in multiple aspects of the socio-economic space and facilitates interaction between agents within multiple aspects of the economy. The existence of platforms expresses the presence of more sprawling economic interactions between agents embedded in multiple aspects of the socio-economic space. Platforms can, therefore, facilitate economic interaction between aspects of the socio-economic space and, in doing so, can potentially broker these relations.

Part~\ref{part:entrepreneurshipPlatformEconomy} formally illustrates the notion of aspects and platforms---and the interaction between these notions---through the use of hypergraphs and hyperlinks. Empirically, we illustrate aspects, platforms, and entrepreneurship within this environment with an analysis of directorate networks. Of specific interest are the directorate networks of New York City during its development in the early 20th century.

\subsection*{Chapter breakdown}

This Part is comprised of two interlinked chapters. The first, \emph{Measuring control in hypergraphs}, provides a formal investigation into how aspects and platforms are considered in terms of hypergraphs, and how influence and positions of power can be identified and measured in a formal manner.

The second chapter, \emph{Control in the Platform economy: The case of New York City}, uses the theory and measures developed in the previous chapter. A specific application is on the assessmenet of the tangled directorate network within the financial, insurance, and railroad industries of New York City between 1902 and 1912. We show the evolving network of relationships and analyse how individuals and firms attained power within the network by forming relationships with other firms between multiple industries of the economy and, in doing so, bridging structural holes. Further, the final chapter investigates the relationship between the economic performance of a firm and the position of the firm within the directorate network. It is found that firms that orientate themselves to fill a structural hole will tend to have better economic performance. A specific example of this is the Banker's Trust.